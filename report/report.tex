\documentclass{article}
\usepackage[letterpaper, total={7.5in, 9in}]{geometry}
\usepackage{graphicx}
\usepackage{fancyvrb}
\usepackage[colorlinks=true, allcolors=blue]{hyperref}

\title{Project Report \\ \normalsize COMP SCI 3800 - Intro to Operating Systems}
\author{Robert Boerwinkle}
\date{}

\begin{document}

\maketitle{}

\section*{software used}

Not all of these are \textit{strictly} necissary, but all are highly recommended. There are alternatives, but this is my suite. It is chosen partially due to the fact that I am using Debian GNU/Linux. Some software will have to be built from source.

\begin{tabular}{|c|c|c|c|}
  \hline
  Name & Debian Package & Usage & Website
  \hline
  TiLP & \verb|tilp2| & Connecting to calculator & \url{http://lpg.ticalc.org/prj_tilp/index.html}
  \hline
  TilEm & \verb|tilem| & Emulation & \url{http://lpg.ticalc.org/prj_tilem/}
  \hline
  rom8x & N/A, built & Building roms & \url{https://www.ticalc.org/archives/files/fileinfo/373/37341.html}
  \hline
\end{tabular}

\section*{methodology}

A ROM is not critical, but very helpful. To obtain the rom, rom8x was used. This software comes with an extensive readme on its operation. In short, it has 2 assembly programs (for my calculator). They each need to be sent to the calculator (with TiLP), and then run\footnote{with the \verb|Asm(| command}. They generate files which can then be copied back to the pc (again, with TiLP). Then use rom8x to combine them with an OS file from TI to produce a \verb|.rom| file. This ROM can then be run with TilEm.

\end{document}
