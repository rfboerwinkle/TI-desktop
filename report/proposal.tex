\documentclass{article}
\usepackage[letterpaper, total={7.5in, 9in}]{geometry}
\usepackage{graphicx}
\usepackage{fancyvrb}
\usepackage[colorlinks=true, allcolors=blue]{hyperref}

\title{Project Proposal \\ \normalsize COMP SCI 3800 - Intro to Operating Systems}
\author{Robert Boerwinkle}
\date{}

\begin{document}

\maketitle{}

\section*{Goals}

This project aims to write an operating system for the Texas Instruments 84+ Silver Edition calculator. The goal is a simple, multi-purpose, responsive operating system for user interface. This will be achieved through the following parts:

\begin{itemize}
  \item a round-robin scheduler
  \item a filesystem for loading processes
  \item static memory allocation
  \item resource allocation with no hold and wait (to prevent deadlock)
  \item system calls available for interacting with hardware
\end{itemize}

These algorithms were chosen to be the simplist to implement with the smallest overhead. The hardware is capable of much more\footnote{\url{https://github.com/KnightOS}}, so there is plenty of room for future development. The end product will consist of the operating system code and a paper detailing the development process.

\section*{Materials}

The only hardware used will be the calculator itself, a USB cable, and a PC. The following list of software will be used: an emulator/debugger\footnote{\url{http://lpg.ticalc.org/prj_tilem/}}, a z80 assembler\footnote{\url{https://www.nongnu.org/z80asm/}}, a linker program to the physical calculator\footnote{\url{http://lpg.ticalc.org/prj_tilp/index.html}}, a tool for building roms\footnote{\url{https://www.ticalc.org/archives/files/fileinfo/373/37341.html}}, a tool for signing the operating system\footnote{\url{https://github.com/abbrev/rabbitsign}}, and some other miscelanious paging tools\footnote{\url{https://www.ticalc.org/archives/files/fileinfo/350/35057.html}}. Signing the operating system also requires keys\footnote{\url{https://brandonw.net/calcstuff/keys.zip}}. The operating system will be built on some basic assembly subroutines specific to the hardware\footnote{\url{https://www.cemetech.net/downloads/files/629/x629}}. The final material is liberal help from online forums\footnote{\url{https://www.cemetech.net/}}.

\end{document}
